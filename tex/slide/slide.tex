\documentclass[xcolor ={table,usenames,dvipsnames}]{beamer}
\usepackage[italian]{babel}
\usepackage{color}
\usepackage{txfonts}
\PassOptionsToPackage{dvipsnames}{xcolor}
\title{Duck Typing in Python}
\author{Author: Tommaso Puccetti}
\institute{Universit\`a  degli Studi di Firenze}
\date{21/12/2018}
%\usepackage{sansmathaccent}
\usetheme{Berlin} 
\useinnertheme{rounded}
\useoutertheme{miniframes} 
\setbeamercovered{dynamic}
\theoremstyle{definition}
\newtheorem{definizione}{Definizione}
\usepackage{tikz}
\usetikzlibrary{arrows}
\usepackage{subfigure}
\usepackage[procnames]{listings}
\usepackage{color}
\usepackage{jlcode}

\definecolor{keywords}{RGB}{255,0,90}
\definecolor{comments}{RGB}{0,0,113}
\definecolor{red}{RGB}{160,0,0}
\definecolor{green}{RGB}{0,150,0}

\lstset{language=Python, 
	backgroundcolor=\color{yellow},
	basicstyle=\fontsize{2}{4}\selectfont\ttfamily\scriptsize, 
	keywordstyle=\color{keywords},
	commentstyle=\color{comments},
	stringstyle=\color{green},
	showstringspaces=false,
	identifierstyle=\color{black},
	procnamekeys={def,class},
}

\begin{document}
	
	\begin{frame}
		\maketitle
			\begin{figure}[h!]
			\centering
			\includegraphics[scale=2]{img/cartoonduck.jpg}
			\label{Interfacce di un CS}
		\end{figure}
	\end{frame}

	\begin{frame}
		\frametitle{Index}
		\begin{enumerate}
			\item Introduction
			\item Python evaluation strategy
			\item Type-checking
			\item Python type-checking
			 \begin{itemize}
					\item Dynamic typing
					\item Strong typing
					\item Annotations
				  \end{itemize}
			\item Object oriented
			\item Duck typing
				\begin{itemize}
				  	\item Main idea
				  	\item Duck typing vs. Static typing
				  \end{itemize}
			\item Conclusion
		\end{enumerate}
	\end{frame}

	\begin{frame}
		\frametitle{Introduction}
		Python is an \textit{\textbf{interpreted}}, \textit{\textbf{multi-paradigm}} language. It was initially designed by Guido van Rossum in 1991 and developed by Python Software Foundation. It supports:
		\begin{itemize}
			\item \textbf{Functional programming };
			\item \textbf{Procedural programming};
			\item \textbf{Objected oriented}.
		\end{itemize}
		\begin{figure}[]
			\centering
			\includegraphics[scale=0.3]{img/python.png}
			\label{Interfacce di un CS}
		\end{figure}
	\end{frame}

	\begin{frame}
		\frametitle{Python evaluation strategy}
			Could be useful to first recall the difference between \textit{\textbf{strict}} and \textit{\textbf{lazy}} evaluation:
			\begin{enumerate}
				\item \textbf{Strict evaluation strategy}: the arguments of a function are fully evaluated to values before evaluating the function call (call by value);
				\item \textbf{Non-strict or Lazy evaluation:} arguments are evaluated only if it is needed in the function body (\textit{call by name})
			\end{enumerate}
			\textbf{Python}:		
			\begin{itemize}
				\item implements \textbf{strict evaluation}.
			\end{itemize}
	\end{frame}

	\begin{frame}[fragile]
		\frametitle{Strict evaluation: example}
		In Python we never get \textit{true} because  it force the evaluation of the function wich contains an infinite loop in the body:
		\begin{lstlisting}
			def infiniteLoop(x):
			    while True:
		           print("do something with x")
		        return x
				
			5 in [5, 10, infiniteLoop(5)]
		\end{lstlisting}
		If we write the same code in \textbf{haskell} we get the \textit{true} value:
		\begin{lstlisting}
			elem 5 [5, 10, infiniteLoop 5]
		\end{lstlisting}
	\end{frame}

	\begin{frame}
		\frametitle{Type-checking }
			\textbf{Type checking} is the process of verifying and enforces the typing rules of a language.
		\begin{enumerate}
			\item \textbf{Dynamic vs. Static}
			\item \textbf{Weak vs. Strong}.
		\end{enumerate}
		\begin{figure}[h!]
			\centering
			\includegraphics[scale=0.14]{img/classification.png}
		\end{figure}
	\end{frame}

	\begin{frame}
		\frametitle{Type-checking: static  }
		In \textbf{statically-typed languages} type-checking is done at compile time.
		\begin{itemize}
			\item \textbf{Advantages}:\begin{itemize}
				\item There is a formal proof of \textbf{type-safety} wich guarantee the absence of run-time errors.;
				\item Static typing guide code development;
				\item Types could be seen as documentation for the code.
			\end{itemize}
			\item \textbf{Disadvantages}: 
			\begin{itemize}
				\item Static typing is a constraint on the program structure;
				\item More code.
			\end{itemize}
		\end{itemize}
	\end{frame}

\begin{frame}
	\frametitle{Type-checking: dynamic}
	In reverse, in \textbf{dynamically-typed languages}
	type checking take place at run-time, during the execution of the code.
	\begin{itemize}
		\item \textbf{Advantages}:
		\begin{itemize}
			\item These languages are more flexible;
			\item less code.
		\end{itemize}
		\item \textbf{Disadvantages}:
		\begin{itemize}
			\item Programs can fail at runtime due to type errors.
			\item It forces runtime checks to occur for every execution of the program. The result is \textbf{less optimized} code.
		\end{itemize}
	\end{itemize}
 \end{frame}

\begin{frame}
	\frametitle{Type-checking: strong and weak}
	\begin{enumerate}
		\item In a \textbf{strongly typed} language every sub-term must be well typed, wich means that the language  enforces strict restrictions on mixing of values with differing data types in an expression.
		\item In a \textbf{Weakly typed} language there are no such restriction.
	\end{enumerate}
	
\end{frame}

	\begin{frame}
		\frametitle{Python type-checking}
			\begin{enumerate}
				\item Python type-checking is \textbf{dynamic}: 
				\begin{itemize}
					\item \textbf{Variables are simply names pointing to objects}, only the object that a variable references has a type;
					\item objects have a type but it is determined at runtime;
					\item variables are not explicitly typed;
					
				\end{itemize}
				\item Python is also \textbf{strongly typed}.
			\end{enumerate}
			
			Let's see the implications by some example.
	\end{frame}

	\begin{frame}[fragile]
		\frametitle{Python dynamic typing example (1)}
			\begin{lstlisting} 
			if False:
				print(10+"ten") 
			else:
				print(10+10)
			\end{lstlisting}
			The first branch never execute, so the type checking ignore the type incongruency.
			
			If we try to execute \textbf{separately} the first branch, the type check raise a type error:
			
			\begin{lstlisting}[keywordstyle=\color{black},
			commentstyle=\color{black},
			stringstyle=\color{black}.]
			TypeError: unsupported operand type(s) for +: 'int' and 'str'
			\end{lstlisting}
	\end{frame}

	\begin{frame}[fragile]
		\frametitle{Python dynamic typing example (2)}
		Another consequnce is that programmers are \textbf{free to bind the same names (variables) to different objects, having  different types}. Then the following statements are perfectly legal:
		
		\begin{lstlisting}
		variable = 10
		variable = "ten"
		\end{lstlisting}
		
		So long as you only perform operations valid for the type the interpreter doesn't care what type they actually are. 
	\end{frame}

	\begin{frame}[fragile]
		\frametitle{Python strong typing example}
		Python is not allowed to perform operations inappropriate to the type of the object: 
		\begin{lstlisting}
		print(10 + "ten")
		\end{lstlisting}
		
		In a \textbf{weakly-typed} language, like PHP, the integer is forced to be a string and no type error is raised:
		\begin{lstlisting}
		$temp = "ten"; 
		$temp = $temp + 10; // no error caused
		echo $temp;
		\end{lstlisting}
	The output will be "ten10".
	\end{frame}

%\begin{frame}[fragile]
%		\frametitle{Some exceptions (1)}
%		There are some operations allowed even in case of type incongruence.\\
%		The\textbf{ boolean equivalence} is permitted in Python 2 and 3: 
%		
%		\begin{lstlisting}
%			print("10" == 10)
%			print("10" != 10)
%		\end{lstlisting}
%		Returning:
%		\begin{lstlisting}
%			False
%			True
%		\end{lstlisting}
%	\end{frame}

%	\begin{frame}[fragile]
%		\frametitle{Some exceptions (2)}
%		In Python 2 "\textit{grather than}" and \textit{"less than"} are permitted:
%		
%		\begin{lstlisting}
%		print("10">10)
%		print("10"<10)
%		\end{lstlisting}
%		
%		Returning:
		
%		\begin{lstlisting}
%		True
%		False
%		\end{lstlisting}
%		
%		Python 3 do not allowed to do "\textit{grather than}" and \textit{"less than"} controls like these.\\	
%	\end{frame}

	\begin{frame}[fragile]
		\frametitle{Annotations}
			Annotations were introduced in Python 3.0 and are the main way to add type hints to the code. We can annotate both \textbf{function} and \textbf{variable}.
			
			\begin{lstlisting}
				import math
				
				pi: float = 3.142
				
				def circumference(radius: float) -> float:
					return 2 * math.pi * radius
			\end{lstlisting}
			
			Type hints and annotations \textbf{\textit{do not add a real static typechecking}} in native Python, so they do not effect the code performance.\\
	\end{frame}

	\begin{frame}
		\frametitle{Annotations: why use it?}
		\textbf{From PEP 484}:\\
		\textit{" <...>using type hints for performance optimizations is left as an exercise for the reader"}.\\
		
		\textbf{Advantages:}
		\begin{itemize}
			\item Type hints help document your code;
			\item Type hints improve IDEs and linters. This allows IDEs to offer better code completion and similar features.
		\end{itemize}
		\textbf{Disadvantages}
		\begin{itemize}
			\item Type hints take developer time and effort to add.
			\item Type hints introduce a slight penalty in start-up time.
		\end{itemize}
	\end{frame}

	\begin{frame}[fragile]
		\frametitle{Object oriented}
	\begin{lstlisting}
		class Duck():
			def __init__(self, name, colour):
				self.name = name
				self.colour = colour
			def quack(self):
				return "Quaaack"
			def fly(self):
				return "The duck is flying"
		
		donald = Duck("Donald","white")
				
		donald.name
		donald.colour
		donald.quack()
		donald.fly()
		\end{lstlisting}
	\end{frame}

	\begin{frame}[fragile]
		\frametitle{Object oriented: self}
		\begin{itemize}
			\item The first argument of every class method is always a reference to the current instance of the class (\textit{\textbf{self}}).
			\item The \textbf{\textit{self}} world is the equivalent of \textbf{\textit{this}} in \textbf{Java}. However Java do not requires to pass \textit{this} explicitly as a first parameter of a method: it could be used straight in the body.
			\item However self\textbf{ is not a reserved keyword} in Python, is just a strong convention.
		\end{itemize}
	%	\begin{lstlisting}[basicstyle=\fontsize{2}{4}\selectfont\ttfamily\tiny]
	%	class Duck():
	%		def __init__(myself, name, colour):
	%			myself.name = name
	%			myself.colour = colour
	%		def quack(myself):
	%			return "Quaaack"	
	%		def fly(myself):
	%			return "The duck is flying"
	%	\end{lstlisting}
	\end{frame}

	%\begin{frame}[fragile]
		%\frametitle{Object oriented (3)}
		%In Python \textbf{is not possible to define multiple constructor} for a class, still is possible
		%to define a default value if one is not passed.
	%	
	%	\begin{lstlisting}
	%	class Parrot():
	%		def __init__(self, name = "Perry"):
	%			self.name = name
	%	
	%	bird1 = Parrot()
	%	bird2 = Parrot("Jack")
	%	
	%	print(bird1.name)
	%	print(bird2.name)	\end{lstlisting}
	%	The output would be:\\
	%	"Perry"\\
	%	"Jack"
	%\end{frame}
	
%	\begin{frame}[fragile]
%		\frametitle{Object oriented: inheritance and polymorphism}
%		\begin{lstlisting}[basicstyle=\fontsize{2}{4}\selectfont\ttfamily\tiny]
%	class Person():
%		def __init__(self, name, surname):
%			self.name = name
%%		def get_info(self):
%			return self.name + " " + self. surname
%	
%	class Teacher(Person):
%		def __init__(self, name, surname):
%			Person.__init__(self, name, surname)
%		def get_info(self):
%			return self.name + " " + self. surname + " is a teacher"
	
%	class Student(Person):
%		def __init__(self, name, surname, id):
%			Person.__init__(self, name, surname)
%			self.id = id
%		def get_info(self):
%			return self.name + " " + self. surname + " is a student"
%		def get_id(self):
%			return self.id
%		\end{lstlisting}
%	\end{frame}
	
	

	\begin{frame}
		\frametitle{Duck typing}
		\textit{ If it looks like a duck, swims like a duck, and quacks like a duck, then it probably is a duck.}\\
		\begin{figure}[h!]
			\centering
			\includegraphics[scale=0.75]{img/duck.png}
			\label{Interfacce di un CS}
		\end{figure}
	\end{frame}

	\begin{frame}
		\frametitle{Duck typing: main idea}
		\textbf{Duck typing} is a concept related to dynamic typing in an object oriented language:
		\begin{itemize}
			\item Implementing duck typing the type of an object is not checked at runtime. Instead the type checker looks for the presence of a given method or attribute.
			\item \textbf{The idea is that it doesn't actually matter what type my data is - just if i can do with it what i want.}
			\item Duck typing implements polymorphism in Python.
		\end{itemize}
	\end{frame}

	\begin{frame}[fragile]
		\frametitle{Duck typing: example (1)}
		%basicstyle=\fontsize{2}{4}\selectfont\ttfamily\tiny
		%class Bird():
		%def fly():
		%pass
		\begin{lstlisting}[]
		
		class Duck(Bird):
			def quack(self):
				return "Quaaack"
			def fly(self):
				return "The duck is flying"
		
		class Parrot(Bird):
			def quack(self):
				return "The parrot imitates a quack"
			def fly(self):
				return "The parrot is flying"
		
		class Man():
			def quack(self):
				return "The man imitates a quack too"

		\end{lstlisting}
	\end{frame}

	\begin{frame}[fragile]
		\frametitle{Duck typing: example (2)}
		\begin{lstlisting}	
		v = [Duck(), Parrot(), Man()]
		
		for i in v:
			print(i.fly())
		\end{lstlisting}
		If we try to use the \textit{fly()} method  over the entire collection of objects an error is raised at runtime:
		\begin{lstlisting}[keywordstyle=\color{black},
		commentstyle=\color{black},
		stringstyle=\color{black}.]	
	The duck is flying
	The parrot is flying
	Traceback (most recent call last):
	File "/home/tommaso/git/ducktyping-tpl/code/ducklist.py", 
	line 23, in <module> print(i.fly())
	AttributeError: Man instance has no attribute 'fly'
		\end{lstlisting}
	\end{frame}

	\begin{frame}[fragile]
		\frametitle{Duck typing: example (3)}
		\begin{lstlisting}			
		for i in v:
			print(i.quack())
		\end{lstlisting}
		Even if the istance of Man is not a subtype of the Bird class the type-checker do not raise any type error. The output would be:
		
		\begin{lstlisting}[keywordstyle=\color{black},
		commentstyle=\color{black},
		stringstyle=\color{black}.]	
		Quaaack
		The parrot imitates a quack
		The man imitates a quack too
		\end{lstlisting}
		
		
		
	\end{frame}

	

	\begin{frame}[fragile]
		\frametitle{Duck typing vs Static typing (1)}
		\begin{lstlisting}
		class Car:
			def __init__(self, engine):
				self.engine = engine
			def run(self):
				self.engine.turn_on()			
		\end{lstlisting}
		
		\begin{itemize}
			\item This is a classical example of \textbf{dependency injection};
			\item Note that my Car does not depends on any concrete implementation of engine: i'm just using a dependency injected instance of something that responds to a \textit{turn\_on} message;
			\item I could say my class Car depends on an interface. But I did not have to declare it. %\textbf{It is an "automatic interface".}\\
		\end{itemize}	
	\end{frame}
	
	\begin{frame}[fragile]
		\frametitle{Duck typing vs Static typing (2)}
		\begin{lstlisting}
			class ElectricEngine:
				def turn_on(self):
					return "zzzz"
			
			class EngineV8:
				def turn_on(self):
					return "brooom"
			
			electric = ElectricEngine()
			fueled = EngineV8()
			
			car1 = Car(fueled)
			car2 = Car(electric)
			print(car1.run())
			print(car2.run())
			
			zzzz
			brooom
		\end{lstlisting}
	\end{frame}
	
	\begin{frame}
		\frametitle{Duck typing vs Static typing (3)}
		In statically-typed language, like Java, if we want to pass different objects to the Car contructor, they has to be for example \textbf{different implementation of a common interface}:
		\begin{itemize}
			\item we had to declare explicit an interface (\textit{IEngine for example});
			\item declare its implementation (EngineV8 or ElectricEngine);
			\item explicitly define my Car parameter to be an implementation of IEngine.
		\end{itemize} 
		
	\end{frame}
	
	\begin{frame}[fragile]
		\frametitle{Duck typing vs Static typing (3)}	
		\begin{lstlisting}
		interface IEngine {
			void turnOn();
		}
		
		public class EngineV8 implements IEngine {
			public void turnOn() {
				// do something here
		}}
		
		public class Car {
			public Car(IEngine engine) {
				this.engine = engine;
				}
			public void run() {
				this.engine.turnOn();
				}
		}
		\end{lstlisting}
	\end{frame}

%	\begin{frame}[fragile]
%		\frametitle{Add method to a class}
	%		In Python there is a difference between:
	%	\begin{itemize}
	%		\item \textbf{Function};
	%		\item \textbf{Bound method}.
	%	\end{itemize}
		
		
	%	\begin{lstlisting}
	%	>>> def foo():
	%	...     print "foo"
	%	...
	%	>>> class A:
	%	...     def bar( self ):
	%	...         print "bar"
	%	...
	%	>>> a = A()
	%	>>> foo
	%	
	%	<function foo at 0x00A98D70>
	%	>>> a.bar
	%	<bound method A.bar of <__main__.A instance at 0x00A9BC88>>
	%	\end{lstlisting}
		
%	\end{frame}
	
	\begin{frame}[fragile]
		\frametitle{Add method to a class}
		\begin{itemize}
			\item Due to the flexibility given by Duck typing is possible to add a function to a class, \textbf{at run-time}.
			\item In a \textbf{statically-typed} languages this behaviour is \textbf{not possible: }once a class is loaded by the classloader there's no way to modify the code.
		\end{itemize}
	\end{frame}

	\begin{frame}[fragile]
		\frametitle{Add method to a class: example (1)}
		\begin{lstlisting}	
		class Duck(Bird):
			def quack(self):
				return "Quaaack"
			def fly(self):
				return "The duck is flying"
		
		class Parrot(Bird):
			def quack(self):
				return "The parrot imitates a quack"
			def fly(self):
				return "The parrot is flying"
				
		class Man():
			def __init__(self, name):
				self.name = name
			def quack(self):
				return "The man imitates a quack too"
		\end{lstlisting}
		
	\end{frame}
	
	\begin{frame}[fragile]
		\frametitle{Add method to a class: example (2)}
		\begin{lstlisting}
		donald = Duck()
		charlie = Parrot()
		john = Man("John")
		jack = Man("Jack")
		
		v = [donald, charlie, john, jack]
		
		def fly(self):
			return "Takes a plane"
		
		Man.fly = fly
		
		
		\end{lstlisting}
	\end{frame}

	\begin{frame}[fragile]
		\frametitle{Add method to a class: example (3)}
		\begin{lstlisting}
			for i in v:
				print(i.fly())
		\end{lstlisting}
		Every istance of the Man class, even if previously instantiated, now has the fly method.
		\begin{lstlisting}[keywordstyle=\color{black},
		commentstyle=\color{black},
		stringstyle=\color{black}.]	
		The duck is flying
		The parrot is flying
		John takes a plane
		Jack takes a plane
		\end{lstlisting}
	\end{frame}

	\begin{frame}[fragile]
		\frametitle{Add method to a single instance of a class (1)}
			It is possible to add a method to a single istance of a class but we have a problem: \textbf{the function is not automatically bound} when it's attached directly to an istance.
			
				\begin{lstlisting}
			john.fly = fly
			
			for i in v:
				print(i.fly())
				
   The duck is flying
   The parrot is flying
   Traceback (most recent call last):
   File "/home/tommaso/git/ducktyping-tpl/code/pokelist.py", line 33, 
   in <module> print(i.fly())
   TypeError: fly() takes exactly 1 argument (0 given)
			\end{lstlisting}
	\end{frame}
	
	\begin{frame}[fragile]
		\frametitle{Add method to a single instance of a class (2)}
			
		To properly bound the method only to "john" istance we had to use the module \textit{types}:
		
		\begin{lstlisting}
		import types
		
		john.fly = types.MethodType(fly, john)
		
		for i in v:
			print(i.fly())
		\end{lstlisting}
	\end{frame}

	\begin{frame}[fragile]
		\frametitle{Add method to a single instance of a class (3)}
			We still have an error but this time is caused by the istance "jack", proving that we added the method fly only to one istance of Man. 
		
		\begin{lstlisting}[keywordstyle=\color{black},
		commentstyle=\color{black},
		stringstyle=\color{black}.]
	The duck is flying
	The parrot is flying
	John takes a plane
	Traceback (most recent call last):
	File "/home/tommaso/git/ducktyping-tpl/code/pokelist.py",
	line 35, in <module>
	print(i.fly())
	AttributeError: Man instance has no attribute 'fly'
		\end{lstlisting}
	\end{frame}

\begin{frame}
	\frametitle{Conclusion}
	Duck typing is a handy way to have a \textbf{polymorphic} code without relying on abstract classes/interfaces:
	\begin{itemize}
		\item a statement calling a method on an object does not rely on the declared type of the object (at compile time)
		\item The object simply must supply an implementation of the method called, when called, at run-time. 
		\item It gives flexibility to the language.
		\item Doesn't guarantee the absence of type error during run-time.
	\end{itemize}
\end{frame}

\begin{frame}[fragile]
	\begin{figure}[h!]
		\centering
		\includegraphics[scale=0.4]{img/doctor.jpg}
		\label{Interfacce di un CS}
	\end{figure}
\end{frame}




	

		
	
	

	
	
	
	
	
	
	
	
	
	
	
\end{document}